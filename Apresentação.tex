\documentclass{beamer}
\usepackage[T1]{fontenc}
\usepackage[utf8]{inputenc}
\usepackage{lmodern}
%\usepackage[simplified]{pgf-umlcd}
\usepackage{graphicx}
\usepackage{hyperref}
%\usepackage{qrcode}

\title{Aplicação do Padrão de Projeto Facade para Implementação de um Software Auxiliar de Atividades Técnicas em Laboratórios de Química}
\author{Caio Cardoso S. G. Souza - 11814871}

\begin{document}
  \begin{frame}
   \titlepage
  \end{frame}
%\begin{comment}
  \begin{frame}
    \frametitle{Sumário}
    \tableofcontents
  \end{frame}
  
  \begin{frame}
    \section{Padrão Facade}
      \frametitle{Padrão Facade}
      \subsection{Intenção e objetivo}
      \framesubtitle{Intenção e objetivo}
        \begin{itemize}
          \item Fornecer uma interface unificada para o acesso das funcionalidades de um ou mais subsistemas... \textbf{NÃO CONFUNDIR COM INTERFACE DO JAVA}
        \end{itemize}
  \end{frame}
%\end{comment}
%\begin{comment}  
  \begin{frame}
    \frametitle{Padrão Facade}
    \subsection{Motivação}
    \framesubtitle{Motivação}
    \begin{itemize}
      \item Complexidade em sistemas grandes e/ou com diversos subsistemas
      \item Diminuir nível de dificuldade para o usuário para esses sistemas
    \end{itemize}
  \end{frame}
%\end{comment}
%\begin{comment}
  \begin{frame}
    \frametitle{Padrão Facade}
    \subsection{Aplicações}
    \framesubtitle{Aplicações}
    É recomendado o padrão Facade nas seugintes ocasiões:
    \begin{itemize}
      \item Fornecer uma interface simples para um sibsistema complexo;
      \item Existem muitas dependências entre os clientes e as classes de implementação de uma abstração;
      \item Organizar os subsistemas em camadas, onde cada fachada define o ponto de entrada dos subsistemas;
    \end{itemize}  
  \end{frame}
%\end{comment}

  \begin{frame}
    \frametitle{Padrão Facade}
    \subsection{Estrutura}
    \framesubtitle{Estrutura}
    \begin{columns}
      \begin{column}{.5\textwidth}
        Fachada:
        \begin{itemize}
          \item A fachada sabe quais classes do subsistema são responsáveis por determinadas tarefas
          \item Delega as requisições a essas classes
        \end{itemize}
      \end{column}
      \begin{column}{.5\textwidth}
        Classes do subsistema:
        \begin{itemize}
          \item Compõem o subsistema em si
          \item Não possuem conhecimento da fachada
          \item Executam as tarefas requisitadas
        \end{itemize}
      \end{column}
    \end{columns}
  \end{frame}
  
  \begin{frame}
    \subsection{Consequêcias}
    \framesubtitle{Consequências}
    \begin{itemize}
        \item Isolamento e facilidade de uso, protegendo os clientes das componentes do subsistema e reduzindo o número de objetos
        \item Acoplamento fraco do código entre os subsistemas e seus clientes, permitindo que as componentes do subsistema variem sem afetar os clientes
        \item Estruturas em camadas, ajudando a organizar o sistema e eliminando dependências complexas ou circulares
    \end{itemize}
    \frametitle{Padrão Facade}
  \end{frame}
    
    \begin{frame}
      \frametitle{Padrão Facade}
      \framesubtitle{Consequências}
      \begin{itemize}
        \item Minimiza a necessidade de recompilação quando as classes dos subsistemas mudam
        \item A fachada não impede que as aplicações usem classes do subsistema diretamente se precisarem, permitindo que ao aplicar esse padrão o projeto possa se adaptar entre facilidade pela fachada e generalidade por meio do acesso direto
      \end{itemize}
  \end{frame}
  
  \begin{frame}
    \subsection{Implementação}
    \framesubtitle{Implementação}
    Para a implementação do padrão Facade é recomendado:
    \frametitle{Padrão Facade}
    \begin{itemize}
      \item Usar uma classe abstrata como fachada de tal forma que suas classes filhas sejam diferentes implementações do subsistema, reduzindo mais o acoplamento
      \item Costomizar uma fachada com elementos do subsistema, tornando o código mais versátil
    \end{itemize}
  \end{frame}
  
  \begin{frame}
    \frametitle{Padrão Facade}
    \subsection{Exemplo}
    \framesubtitle{Exemplo}
    \begin{figure}
      \centering
      \includegraphics[width=1\textwidth]{codexemple}
    \end{figure}
  \end{frame}
  
  \begin{frame}
    \frametitle{Padrão Facade}
    \subsection{Usos conhecidos}
    \framesubtitle{Usos conhecidos}
    \begin{itemize}
      \item Sistema de compilador ObjectWorks para Smalltalk (exemplo)
      \item Framework ET++
      \item Choices OS
    \end{itemize}
  \end{frame}
  
  \begin{frame}
    \frametitle{Padrão Facade}
    \subsection{Padrões Relacionados}
    \framesubtitle{Padrões Relacionados}
    \begin{itemize}
        \item Singleton
        \item Abstract Factory
        \item Mediator
        \item Adapter
    \end{itemize}
  \end{frame}
  
  %%%PARTE 2%%%
  \begin{frame}
    \section{Aplicação do Facade}
    \frametitle{Aplicação do Facade}
    \framesubtitle{Objetivo da solução}
    Fornecer um software capaz de auxiliar químicos em tarefas de desígnio técnico em um laboratório
    \begin{center}
      \textbf{QUE TAREFAS?}
    \end{center}
  \end{frame}
  
  \begin{frame}
    \frametitle{Aplicação do Facade}
    \subtitle{Serviços}
    \framesubtitle{Serviços}
    \begin{itemize}
      \item Adicionar ficha de resíduo
      \item Remover ficha de resíduo
      \item Consultar fichas de resíduo
      \item Adicionar no estoque
      \item Remover do estoque
      \item Consultar estoque
      \item Purgar bomba do HPLC
      \item Ajustar preção da bomba do HPLC
      \item Checar funcionamento do injetor do HPLC
      \item Checar temperatura do forno do HPLC
      \item Checar módulo central do HPLC
      \item Calibrar detector A do HPLC
      \item Calibrar detector B do HPLC
    \end{itemize}
  \end{frame}
  
  \begin{frame}
    \frametitle{Aplicação do Facade}
    \framesubtitle{Serviços}
    \begin{itemize}  
      \item Calibrar pHmetro
      \item Checar eletrodo do pHmetro
      \item Checar temperatura do pHmetro
      \item Troca de eletrólito do pHmetro
      \item Calibração simples da balança
      \item Calibração completa da balança
      \item Filtrar água do Mili-q
      \item Checar condutância do Mili-q
      \item Checar nível do estoque do Mili-q
      \item Limpeza do sistema Mili-q
      \item Relatórios
    \end{itemize}
  \end{frame}
  
  \begin{frame}
    \subsection{Modelagem}
    \frametitle{Aplicação do Facade}
    \framesubtitle{Modelagem: Diagrama de Comunicação}
    \begin{figure}
      \centering
      \includegraphics[width=1\textwidth]{rascunho}
    \end{figure}
  \end{frame}
  
  \begin{frame}
    \frametitle{Aplicação do Facade}
    \framesubtitle{Modelagem: Diagrama de Comunicação}
    \begin{figure}
      \centering
      \includegraphics[width=0.9\textwidth]{MDC}
    \end{figure}
  \end{frame}
  
  \begin{frame}
    \frametitle{Aplicação do Facade}
    \framesubtitle{Modelagem: Diagrama de Classe}
    \begin{figure}
      \centering
      \includegraphics[width=0.65\textwidth]{MDc}
    \end{figure}
  \end{frame}
  
  \begin{frame}
    \subsection{Algoritmo}
    \frametitle{Aplicação do Facade}
    \framesubtitle{Algoritmo}
    \begin{columns}
      \begin{column}{.5\textwidth}
        \begin{figure}
          \centering
          \includegraphics[width=1\textwidth]{A1}
        \end{figure}
      \end{column}
      \begin{column}{.5\textwidth}
        \begin{figure}
          \centering
          \includegraphics[width=1\textwidth]{A2}
        \end{figure}
      \end{column}
    \end{columns}
  \end{frame}
  
  \begin{frame}
    \section{Referências e Links externos}
    \frametitle{Referências e Links externos}
    \begin{columns}
      \begin{column}{.5\textwidth}
        \textbf{Referências:}
        \begin{itemize}
          \item GAMMA, E. et al. \textbf{Design Patterns: Elements of Reusable Object-Oriented Software}. Boston: Addison-Wesley, 1994.
          \item DR. EDWARD LAVIERI. \textbf{Hands-On Design Patterns With Java;Learn Design Patterns That Enable The Building Of Large-Scale Software Architectures}. S.L.: Packt Publishing, [s.d.]. 
        \end{itemize}
      \end{column}
      \begin{column}{.5\textwidth}
        Link para repositório do projeto no \href{https://github.com/MafiusKity/SAATLQ/tree/main}{\textit{GitHub}}
        \begin{figure}
          \centering
          \includegraphics[width=1\textwidth]{qr}
        \end{figure}
      \end{column}
    \end{columns}
    \begin{center}
      \textbf{OBRIGADO PELA ATENÇÃO!}
    \end{center}
  \end{frame}
\end{document}
